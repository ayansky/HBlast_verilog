\section{Related Works}
\label{sec:background}


Presently the most popular BLAST implementation is that of the National Center for Biotechnology Information (NCBI)~\cite{ncbiBlast}, which runs on the NCBI servers.
The high flexibility and scalability of software implementation and easy to use web-based interface make NCBI implementation very attractive.
However, CPU based server technology is becoming insufficient in meeting the performance requirement due to the rapid growth in database size as well as query sequences.
NCBI has proposed a number of solutions to improve the efficiency of software implementation such as the two-hit method and the Gapped BLAST~\cite{WIENBRANDT20111967}.
In spite of enhancements, the software tools still takes long run-time.
Hardware accelerators have high potential in this regard, since they can provide a high degree of bit-level parallelism.
The proposed implementation is not an alternate for NCBI implementation, but try to complement it with support for hardware acceleration.

FPGAs are utilized in sequence comparison in Large Sequence Databanks since they significantly improve performance due to their parallel architecture. 
There are a number of proposed and applied implementations of BLAST algorithm targeting FPGAs~\cite{oliver2005hyper}. 
Almost all of the implementations are based on one of the two searching strategies, the Word Position Record Based Search (WPRBS) and Systolic arrays~\cite{guo2012systolic}. 
The latter is a relatively new method and not a widely used for implementation of BLAST algorithm. 

In WPRBS technique, only one word can be searched per clock, and consequently, only one hit can be found per clock \cite{guo2012open}. 
Timing issue is also caused by shortage in WPRBS searching capacity due to limited number of memory ports in FPGAs. 
Examples of such architecture are Mercury BLASTn \cite{buhler2007mercury} and Mitrion \cite{guo2012systolic}. 
In these two systems, storage tables are stored in the external memory SRAM that is attached to the FPGA, since the tables storing the words of long query sequences will require vast amount of space in internal memory RAM. 
Searching is performed by comparing one subject sequence from database to every word in the storage table simultaneously during one clock cycle. 
It is clear that performance of the system will be low since accessing external memory SRAM for getting subject sequence from database every time consumes significant amount of time. 

By addressing the above timing issue, FPGA/FLASH \cite{lavenier2007reconfigurable} could achieve better results by minimizing the number of accesses to the external memory. 
This is done by enabling detection of several exact matches/hits simultaneously. 
This is due to the index structure of the database, where each word is associated with its position in the database and neighboring elements, hence, avoiding random accesses to database \cite{guo2012systolic}. 
However, large memory space is needed to store index of the database as well as database itself. 
For database exponentially growing every year, this architecture will not be efficient in terms of storage capacity. 
In order to achieve better searching results, the architecture Multiengines BLASTn \cite{sotiriades2007design} was proposed. 
It is still based on WRPBS, but designed in such way that compares 64 subject sequences simultaneously due to its 64 identical computing units. 
As all of the above-mentioned implementations based on WRPBS, they suffer from timing and memory limitations. 
Further these architectures are not available as open source implementations preventing their wider adoption.

There are a few projects which are available as open source. 
RC-BLAST~\cite{datta2009} and  systolic array-based architecture~\cite{guo2012open} are two such implementations.
Although, RC-BLAST is stated to outperform a modern's day general purpose PC computer, this implementation is mostly dependent on the database length, query length and the number of hits identified. 
For instance, this system's operation linearly depends on the size of the query, that is if query length increases, the FPGA system's performance rises. 
Also, RC-BLAST's performance drops if there are many hits found, which increases the frequency of memory access.

In another open source implementation \cite{guo2012open}, the researchers present a systolic array-based FPGA parallel architecture for BLAST algorithm. 
This architecture can detect more than one hit per clock cycle making it faster than WPRBS-based architectures. 
The speed advantage of this architecture is obtained by merging two adjacent hits into one and speed up the extension step. 
This implementation manages resource utilization better than Tree-BLAST \cite{herbordt2006single}, performs scanning faster than Mercury BLASTn \cite{buhler2007mercury}, and deals with long sequences more effectively than Mercury BLASTp \cite{harris2007banded}. 
However, the architecture focuses only on the first two steps of BLAST algorithm, finding the hit and ungapped expansion. 

HBlast is a complete system-level implementation of BLAST targeting both on-premises as well as cloud-based acceleration.
The implementation supports latest off-the-self FPGA boards as well as FPGA instances in the cloud.
The support for cloud-based FPGA instances enables users dynamically scale the system based on their performance-cost constraints.
The implementation can detect multiple word hits simultaneously due to the parallel implementation of the detection array.
The query sequence is stored in the FPGA-internal memory and the database in the high-speed external DRAM memory.
This minimizes the number of external memory accesses at the same time supports large database sequences.
